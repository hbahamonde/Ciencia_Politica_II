%----------------------------------------------------------------------------------------
%	PACKAGES AND OTHER DOCUMENT CONFIGURATIONS
%----------------------------------------------------------------------------------------

\documentclass[10pt]{article}
\usepackage{lipsum} % Package to generate dummy text throughout this template

%\usepackage[light, math]{iwona}
%\usepackage[sc]{mathpazo} % Use the Palatino font
\usepackage[T1]{fontenc} % Use 8-bit encoding that has 256 glyphs
\linespread{1.05} % Line spacing - Palatino needs more space between lines
\usepackage{microtype} % Slightly tweak font spacing for aesthetics

\usepackage[hmarginratio=1:1,top=32mm,columnsep=20pt]{geometry} % Document margins
\usepackage{multicol} % Used for the two-column layout of the document
\usepackage[hang, small,labelfont=bf,up,textfont=it,up]{caption} % Custom captions under/above floats in tables or figures
\usepackage{booktabs} % Horizontal rules in tables
\usepackage{float} % Required for tables and figures in the multi-column environment - they need to be placed in specific locations with the [H] (e.g. \begin{table}[H])

\usepackage{lettrine} % The lettrine is the first enlarged letter at the beginning of the text
\usepackage{paralist} % Used for the compactitem environment which makes bullet points with less space between them

\usepackage{abstract} % Allows abstract customization
\renewcommand{\abstractnamefont}{\normalfont\bfseries} % Set the "Abstract" text to bold
\renewcommand{\abstracttextfont}{\normalfont\small\itshape} % Set the abstract itself to small italic text

\usepackage{titlesec} % Allows customization of titles
\renewcommand\thesection{\Roman{section}} % Roman numerals for the sections
\renewcommand\thesubsection{\Roman{subsection}} % Roman numerals for subsections
\titleformat{\section}[block]{\large\scshape\centering}{\thesection.}{1em}{} % Change the look of the section titles
\titleformat{\subsection}[block]{\large}{\thesubsection.}{1em}{} % Change the look of the section titles

\usepackage{fancybox, fancyvrb, calc}
\usepackage[svgnames]{xcolor}


%----------------------------------------------------------------------------------------
%	DOCUMENT ID (Department, Professor, Course, etc.) 
%----------------------------------------------------------------------------------------

\usepackage{fancyhdr} % Headers and footers
\pagestyle{fancy} % All pages have headers and footers
\fancyhead{} % Blank out the default header
\fancyfoot{} % Blank out the default footer
\fancyhead[C]{Ensayo Midterm} % Custom header text
\fancyfoot[RO,LE]{\thepage} % Custom footer text

%----------------------------------------------------------------------------------------
%	MY PACKAGES 
%----------------------------------------------------------------------------------------

\usepackage{amsmath}	
\usepackage[makeroom]{cancel}
%\usepackage{rotating}
\usepackage{textcomp}
\usepackage{caption}
\usepackage{etex}
%\usepackage[export]{adjustbox}
%\usepackage{afterpage}
%\usepackage{filecontents}
\usepackage{color}
\usepackage{latexsym}
\usepackage{lscape}				%\begin{landscape} and \end{landscape}
\usepackage{amsfonts}
%\usepackage{mathabx}
\usepackage{amssymb}
\usepackage{soul} % wrapped underline 
%\usepackage{dashrule}
%\usepackage{txfonts}
%\usepackage{pgfkeys}
%\usepackage{framed}
\usepackage{tree-dvips}
\usepackage{caption}
%\usepackage{fancyvrb}
%\usepackage{pgffor}
\usepackage{xcolor}
%\usepackage{pxfonts}
\usepackage{wasysym}
\usepackage{authblk}
%\usepackage{paracol}
\usepackage{setspace}
%\usepackage{qtree}
%\usepackage{tree-dvips}
\usepackage{sgame}				% shouldn't have neither array nor tabularx packages
\usepackage{tikz}
%\usetikzlibrary{trees}
\usepackage[latin1]{inputenc}
%\label{tab:1} 		%\autoref{tab:1}	%ocupar para citar.
% \hyperlik{table1}	\hypertarget{table1} 
% \textquoteright			%apostrofe
\usepackage{hyperref} 		%desactivar para link rojos





%----------------------------------------------------------------------------------------
%	Other ADDS-ON
%----------------------------------------------------------------------------------------

% independence symbol \independent
\newcommand\independent{\protect\mathpalette{\protect\independenT}{\perp}}
\def\independenT#1#2{\mathrel{\rlap{$#1#2$}\mkern2mu{#1#2}}}


% VERBATIM WITH BACKGROUND COLOR
\newenvironment{colframe}{%
  \begin{Sbox}
    \begin{minipage}
      {\columnwidth%-\leftmargin-\rightmargin-6pt
      }
    }{%
    \end{minipage}
  \end{Sbox}
  \begin{center}
    \colorbox{LightSteelBlue}{\TheSbox}
  \end{center}
}


\hypersetup{
    bookmarks=true,         % show bookmarks bar?
    unicode=false,          % non-Latin characters in Acrobat$'$s bookmarks
    pdftoolbar=true,        % show Acrobat$'$s toolbar?
    pdfmenubar=true,        % show Acrobat$'$s menu?
    pdffitwindow=false,     % window fit to page when opened
    pdfstartview={FitH},    % fits the width of the page to the window
    pdftitle={My title},    % title
    pdfauthor={Author},     % author
    pdfsubject={Subject},   % subject of the document
    pdfcreator={Creator},   % creator of the document
    pdfproducer={Producer}, % producer of the document
    pdfkeywords={keyword1} {key2} {key3}, % list of keywords
    pdfnewwindow=true,      % links in new window
    colorlinks=true,       % false: boxed links; true: colored links
    linkcolor=ForestGreen,          % color of internal links (change box color with linkbordercolor)
    citecolor=ForestGreen,        % color of links to bibliography
    filecolor=ForestGreen,      % color of file links
    urlcolor=ForestGreen           % color of external links
}


% PROPOSITIONS
\newtheorem{proposition}{Proposition}

%\linespread{1.5}

\usepackage[american]{babel}
\usepackage{csquotes}
\usepackage[backend=biber,style=authoryear,dashed=false,doi=false,isbn=false,url=false,arxiv=false]{biblatex}
%\DeclareLanguageMapping{american}{american-apa}
\addbibresource{/Users/hectorbahamonde/Bibliografia_PoliSci/library.bib} 
\addbibresource{/Users/hectorbahamonde/Bibliografia_PoliSci/Bahamonde_BibTex2013.bib} 


%----------------------------------------------------------------------------------------
%	TITLE SECTION
%----------------------------------------------------------------------------------------

%\title{\vspace{-15mm}\fontsize{18pt}{7pt}\selectfont\textbf{Experimental Economists and Psychologists: Two Worlds Apart}} % Article title

%\author[1]{
%\large
%\textsc{H\'ector Bahamonde}\\ 
%\thanks{}
%\normalsize Political Science Dpt. $\bullet$ Rutgers University \\ % Your institution
%\normalsize \texttt{e:}\href{mailto:hector.bahamonde@rutgers.edu}{\texttt{hector.bahamonde@rutgers.edu}}\\
%\normalsize \texttt{w:}\href{http://www.hectorbahamonde.com}{\texttt{www.hectorbahamonde.com}}
%\vspace{-5mm}
%}
%\date{\today}

%----------------------------------------------------------------------------------------

\begin{document}

%\maketitle % Insert title


\thispagestyle{fancy} % All pages have headers and footers

%----------------------------------------------------------------------------------------
%	ABSTRACT
%----------------------------------------------------------------------------------------

%\begin{abstract}
%	ABSTRACT
%\end{abstract}


%----------------------------------------------------------------------------------------
%	CONTENT
%----------------------------------------------------------------------------------------

%\graphicspath{
%{/Users/hectorbahamonde/RU/Term5/Experiments_Redlawsk/Experiment/Data/}
%}
\hspace{-5mm}{\bf Profesor}: H\'ector Bahamonde, PhD.\\
\texttt{e:}\href{mailto:hector.bahamonde@uoh.cl}{\texttt{hector.bahamonde@uoh.cl}}\\
\texttt{w:}\href{http://www.hectorbahamonde.com}{\texttt{www.hectorbahamonde.com}}\\
{\bf Curso}: Ciencia Pol\'itica II.\\
\hspace{-5mm}{\bf TA}: Gonzalo Barr\'ia.

\subsection*{{\bf Instrucciones}}

Este es el temario para el ensayo \emph{midterm}. De las tres preguntas, debes escoger una. Debes escribir tu ensayo con una extensi\'on m\'inima de 1500 palabras (aproximadamente 3 p\'aginas) y una extensi\'on m\'axima de 2000 palabras (aproximadamente 4 p\'aginas). Tu respuesta debe tener formato ensayo: debe ser una an\'alisis cr\'itico de las lecturas. Si te limitas a resumir el texto, tendr\'as una nota deficiente. La cantidad de palabras incluye las citas, pero excluye el t\'itulo y la bibliograf\'ia. {\bf Deber\'as subir tu respuesta a uCampus no antes del plazo estipulado en la misma plataforma en la secci\'on \emph{Tareas}}. Tanto el ayudante como el profesor estar\'an disponibles durante horas y d\'ias laborales para responder preguntas. {\bf Los trabajos atrasados tendr\'an la nota m\'inima, sin excepci\'on---si tienes mala conexi\'on de Internet, planifica tu trabajo con mucha anticipaci\'on, y sube tu trabajo lo antes posible}. Revisa tu trabajo al menos dos veces, y preoc\'upate de citar adecuadamente: {\bf el plagio significa nota m\'inima}. \ul{Tu respuesta debe citar cada trabajo nombrado en la pregunta una vez como m\'aximo \emph{y} como m\'inimo (con una extensi\'on m\'axima de una oraci\'on de dos l\'ineas por cita)}. El \'unico tipo de archivo aceptado es un archivo de texto con extensi\'on \texttt{doc} o \texttt{docx}. Recuerda incluir una seccion ``Bibliograf\'ia'' al final de tu trabajo (esto no cuenta en la cantidad de palabras).

\subsection*{{\bf Escoge una pregunta:}}

\begin{enumerate}
  \item \textcite{Sokoloff:2000ug} y \textcite{Acemoglu:2002uh} explican el rol de las instituciones en la construcci\'on del estado en Am\'erica Latina. En media p\'agina explica ambas teor\'ias. Ahora, por un lado, algunos autores ponen \'enfasis en el periodo colonial/pre-colonial \parencite{Haber1991,Mahoney:2010aa} mientras que otros ponen \'enfasis en el periodo \emph{post}-colonial \parencite{Kurtz:2013aa,Soifer2015a}. Explica en media p\'agina de qu\'e se tratan ambos enfoques. En el resto de tu ensayo toma una posici\'on respecto de qu\'e \emph{timing}---pre-colonial/colonial o post-colonial---responde mejor a los distintos niveles de desarrollo pol\'itico-ec\'onomico en las Am\'ericas que vemos hoy.

  \item Las ``preparaciones para la guerra'' \parencite{Tilly1985} y el \emph{robo organizado} \parencite{Olson1993} parecen ser condiciones necesarias para explicar la formaci\'on del estado moderno. En una p\'agina explica c\'omo el elemento \emph{violencia organizada} presente en ambas teor\'ias conecta (o no) con la recolecci\'on de impuestos directos. En lo que sigue de tu ensayo critica este enfoque (llamado ``belicista'') a la luz de \textcite{Strayer1970} y \textcite{Centeno1997}, y adopta una posici\'on respecto de si las teor\'ias ``belicistas'' nos ayudan (o no) a explicar el origen del estado Latino Americano.

  \item Un componente importante de los estados es su ``poder infraestructural'' \parencite{Mann2008a} ya que permite ``transmitir'' (i.e. \emph{broadcast}) el poder \parencite{Herbst2015}. En una p\'agina explica por qu\'e la dimensi\'on espacial/territorial es tan importante (o no). No olvides mencionar ejemplos concretos abordados en las lecturas mencionadas. En el resto del ensayo, elabora un argumento en el que convenzas a tu lector de por qu\'e los estados no siempre est\'an interesados en ``transmitir'' ese poder a lo largo y ancho del territorio. Ap\'oyate en \textcite{Brinks2020}. En esta parte incorpora un breve ejemplo (un p\'arrafo) de la realidad chilena.

\end{enumerate}





\newpage
\pagenumbering{Roman}
\setcounter{page}{1}
\printbibliography


\end{document}

